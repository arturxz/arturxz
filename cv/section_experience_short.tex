\sectionTitle{Experiência Profissional}{\faSuitcase}
%\renewcommand{\labelitemi}{$\bullet$}
\begin{experiences}
    \emptySeparator
    \experience
        {Janeiro 2019} {Desenvolvedor Júnior | Equipe de Documentos Fiscais Eletrônicos}{Secretaria de Estado da Fazenda de Sergipe (SEFAZ-SE)}{Aracaju-SE}
        {Atual} {
                    \begin{itemize}
                        \item Recepção, processamento e armazenamento de documentos e declarações fiscais eletrônicos (NF-e, NFC-e, BP-e, CT-e, MDF-e, DASN, DASMEI, DIMP, etc.);
                        \item Manutenção em \textit{back-end} e \textit{front-end} de sistemas de recepção, processamento e armazenamento de documentos e declarações fiscais eletrônicos;
                        \item Desenvolvimento  e manutenção \textit{back-end} e \textit{front-end} de sistemas para consulta de dados e fornecimento de documentos para contribuintes e auditores fiscais;
                    \end{itemize}
                }
                {Windows, Java, Eclipse, iWorkplace, SOAP, SoapUI, PLSQL, SQL Developer}
    \emptySeparator
    \experience
        {Novembro 2017} {Pesquisador | Pesquisa, análise e síntese da \textit{Home Care} em Sergipe}{Universidade Federal de Sergipe}{São Cristóvão-SE}
        {Julho 2018}    {
                          \begin{itemize}
                            \item Prospecção de empresas que ofereçam o serviço de \textit{Home Care} em Sergipe;
                            \item Prospecção de \textit{startups} e aplicativos para dispositivos móveis que auxiliem cuidadores ou pacientes em \textit{Home Care};
                            \item Pesquisa exploratória por meio de \textit{survey} sobre o oferecimento de aplicativos que auxiliem cuidadores ou pacientes em \textit{Home Care};
                            \item Escrita de trabalho, aceito no EATIS 2018 com Qualis B3 e apresentado como pôster.
                          \end{itemize}
                        }
                        {}
%    %\experience
%        {Janeiro 2017} {Pesquisador | Gerenciamento de redes para \textit{e-science}}{Universidade Federal de Sergipe}{São Cristóvão-SE}
%       {Julho 2017}
%                {
%                        \begin{itemize}
%                            \item Pesquisa por modelos de rede que suportem alto fluxo de dados (encontrado DMZ Científica);
%                            \item Desenvolvimento de ferramenta para verificação de fluxo de dados com BashScript e iPerf;
%                            \item Prospecção por ferramentas de monitoramento de rede (PerfSONAR);
%                            \item Simulação do fluxo de uma rede para comparação entre o modelo atual da UFS e o proposto na DMZ Científica.
%    %                    \end{itemize}
%                    }
%                    {BashScript, iPerf, PerfSONAR}
    \emptySeparator
    \experience
    {Abril 2015}       {Desenvolvedor}{Programa de Apoio do Desenvolvimento da Aprendizagem Profissional - PRODAP}{Departamento de Computação, Universidade Federal de Sergipe}
    {Dezembro 2016}
                    {
                        Desenvolvimento de sistema de baixo custo para auxílio na administração do departamento e manutenção dos computadores.
                        \begin{itemize}
                            \item Desenvolvimento e manutenção de repositório de pacotes para arquitetura ArchLinux;
                            \item Desenvolvimento de interface \textit{web} para manipulação do repositório;
                            \item Desenvolvimento da integração entre PHP 7 e BashScript usando Python;
                            \item Desenvolvimento de serviço para Windows com finalidade de gerenciar o login de usuários;
                            \item Utilização da arquitetura CubieTruck como servidor do sistema.
                        \end{itemize}
                    }
                    {ArchLinux, BashScript, PHP 7, Apache Server, MySQL, phpMyAdmin, Python 3, Git}
\end{experiences}
